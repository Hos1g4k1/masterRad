% Format teze zasnovan je na paketu memoir
% http://tug.ctan.org/macros/latex/contrib/memoir/memman.pdf ili
% http://texdoc.net/texmf-dist/doc/latex/memoir/memman.pdf
% 
% Prilikom zadavanja klase memoir, navedenim opcijama se podešava 
% veličina slova (12pt) i jednostrano štampanje (oneside).
% Ove parametre možete menjati samo ako pravite nezvanične verzije
% mastera za privatnu upotrebu (na primer, u b5 varijanti ima smisla 
% smanjiti 
\documentclass[12pt,oneside]{memoir} 

% Paket koji definiše sve specifičnosti master rada Matematičkog fakulteta
\usepackage[latinica]{matfmaster} 
%
% Podrazumevano pismo je ćirilica.
%   Ako koristite pdflatex, a ne xetex, sav latinički tekst na srpskom jeziku
%   treba biti okružen sa \lat{...} ili \begin{latinica}...\end{latinica}.
%
% Opicija [latinica]:
%   ako želite da pišete latiniciom, dodajte opciju "latinica" tj.
%   prethodni paket uključite pomoću: \usepackage[latinica]{matfmaster}.
%   Ako koristite pdflatex, a ne xetex, sav ćirilički tekst treba biti
%   okružen sa \cir{...} ili \begin{cirilica}...\end{cirilica}.
%
% Opcija [biblatex]:
%   ako želite da koristite reference na više jezika i umesto paketa
%   bibtex da koristite BibLaTeX/Biber, dodajte opciju "biblatex" tj.
%   prethodni paket uključite pomoću: \usepackage[biblatex]{matfmaster}
%
% Opcija [b5paper]:
%   ako želite da napravite verziju teze u manjem (b5) formatu, navedite
%   opciju "b5paper", tj. prethodni paket uključite pomoću: 
%   \usepackage[b5paper]{matfmaster}. Tada ima smisla razmisliti o promeni
%   veličine slova (izmenom opcije 12pt na 11pt u \documentclass{memoir}).
%
% Naravno, opcije je moguće kombinovati.
% Npr. \usepackage[b5paper,biblatex]{matfmaster}

% Pomoćni paket koji generiše nasumičan tekst u kojem se javljaju sva slova
% azbuke (nema potrebe koristiti ovo u pravim disertacijama)
\usepackage[latinica]{pangrami}

% Datoteka sa literaturom u BibTex tj. BibLaTeX/Biber formatu
\bib{matfmaster-primer}

% Ime kandidata na srpskom jeziku (u odabranom pismu)
\autor{Lazar Čeliković}
% Naslov teze na srpskom jeziku (u odabranom pismu)
\naslov{Upravljanje razvojem mobilnih aplikacija sa fokusom na performanse i kvalitet}
% Godina u kojoj je teza predana komisiji
\godina{2024}
% Ime i afilijacija mentora (u odabranom pismu)
\mentor{dr Vladimir \textsc{Filipović}, redovan profesor\\ Univerzitet u Beogradu, Matematički fakultet}
% Ime i afilijacija prvog člana komisije (u odabranom pismu)
\komisijaA{dr Aleksandar \textsc{Kartelj}, vanredni profesor\\ Univerzitet u Beogradu, Matematički fakultet}
% Ime i afilijacija drugog člana komisije (u odabranom pismu)
\komisijaB{dr Staša \textsc{Vujučić Stanković}, docent\\ Univerzitet u Beogradu, Matematički fakultet}
% Ime i afilijacija trećeg člana komisije (opciono)
% \komisijaC{}
% Ime i afilijacija četvrtog člana komisije (opciono)
% \komisijaD{}
% Datum odbrane (odkomentarisati narednu liniju i upisati datum odbrane ako je poznat)
% \datumodbrane{}

% Apstrakt na srpskom jeziku (u odabranom pismu)
\apstr{%
Mobilne aplikacije i aplikacije generalno često se razvijaju sa primarnim ciljem da se pokrije što je moguće više funkcionalnosti, dok se koncepti kao što su performanse zanemaruju dok problem ne postane evidentan. U ovom radu se razmatra pristup koji ove koncepte stavlja u središte i implementira ih na samom početku razvojnog ciklusa. Rad prikazuje kako postavljanje sistema za merenje performansi, poslovnih metrika i analizu grešaka olakšava dalji razvoj u kasnijim stadijumima projekta. Rešenje koje ćemo analizirati razvijeno je uz pomoć razvojnog okruženja ReactNative uz podršku RTKQ biblioteke sa menadžovanje stanja podataka. Rad prikazuje detalje implementacije ovih sistema kao i benefite koje dobijamo korišćenjem istih. Pomenuti koncepti biće prikazani i analizirani na primeru mobilne aplikacije za vremensku prognozu.
}

% Ključne reči na srpskom jeziku (u odabranom pismu)
\kljucnereci{mobilna aplikacija, performanse, dogadjaji, računarstvo, programiranje}

\begin{document}
% ==============================================================================
% Uvodni deo teze
\frontmatter
% ==============================================================================
% Naslovna strana
\naslovna
% Strana sa podacima o mentoru i članovima komisije
\komisija
% Strana sa posvetom (u odabranom pismu)
% Strana sa podacima o disertaciji na srpskom jeziku
\apstrakt
% Sadržaj teze
\tableofcontents*

% ==============================================================================
% Glavni deo teze
\mainmatter
% ==============================================================================

% ------------------------------------------------------------------------------
\chapter{Uvod}
% ------------------------------------------------------------------------------
\pangrami

\section{Primeri korišćenja klasičnih \LaTeX{} elemenata}
% Primeri citiranja
Ovo je rečenica u kojoj se javlja citat \cite{PetrovicMikic2015}.
Još jedan citat \cite{GuSh:243}.
% Primeri navodnika
Isprobavamo navodnike: "Rekao je da mu se javimo sutra".
% Primer referisanja na tabelu (koja se javlja kasnije)
U tabeli \ref{tbl:rezultati} koja sledi prikazani su rezultati eksperimenta.
% Primer kraćeg ćiriličkog teksta
{\cir Ово је пример ћириличког текста који се јавља у латиничком документу.}
U ovoj rečenici se javlja jedna reč na {\cir ћирилици}.
% Primer korišćenja fusnota
Iza ove rečenice sledi fusnota.\footnote{Ovo je fusnota.}

% Primer dužeg ćirličkog teksta
\begin{cirilica}
  Ово је мало дужи блок текста исписан ћириличким писмом у оквиру
  латиничког документа. Фијуче ветар у шибљу, леди пасаже и куће иза
  њих и гунђа у оџацима.
\end{cirilica}

% Primer korišćenja tabele
\begin{table}
\centering
\caption{Rezultati}
\label{tbl:rezultati}
\begin{tabular}{c>{\centering}p{2cm}c}
\toprule
1 & 2 & 3\\\midrule
4 & 5 & 6\\\cmidrule(rl){1-2}
7 & 8 & 8\\
\bottomrule
\end{tabular}
\end{table}

% Primer korišćenja slike
\begin{figure}[!ht]
  \centering
  \label{fig:grafikon}
  \includegraphics[width=0.5\textwidth]{graph.png}
  \caption{Grafikon}
\end{figure}


% Primer jednostavnije matematičke formule
Evo i jedan primer matematičke formule: $e^{i\pi} + 1 = 0$. 
% Primer referisanja na sliku
Na slici \ref{fig:grafikon} prikazan je jedan grafikon.

% primer kompleksnije matematičke formule
$$
\int_a^b f(x)\ \mathrm{d}x \ =_{def}\ \lim_{\max{\Delta x_k \rightarrow 0}} \sum_{k=1}^n f(x_k^*)\Delta x_k
$$

% primer referisanja na poglavlja i strane poglavlja
Više detalja biće dato u glavi \ref{chp:razrada} na strani \pageref{chp:razrada}.

% primer liste
Možemo praviti i nabrajanja:
\begin{enumerate}
\item Analiza 1
\item Linearna algebra
\item Analitička geometrija
\item Osnovi programiranja
\end{enumerate}

\pangrami

% ------------------------------------------------------------------------------
\chapter{Pregled tehnologija za razvoj mobilnih aplikacija}
\label{chp:pregledTehnologijaZaRazvojMobilnihAplikacija}

% ------------------------------------------------------------------------------
Mobilne aplikacije su doživele neverovatnu evoluciju od svog početka. Svedoci smo kako su one iz jednostavnih alata za komunikaciju ili zabavu prerastle u složene sisteme koji nam pomažu u svakodnevnom životu. Danas, aplikacije nisu samo način da ostanemo povezani s drugima, već i sredstvo za upravljanje finansijama, učenje novih veština, pa čak i za praćenje zdravlja i fitnessa. Razvoj tehnologija poput veštačke inteligencije i mašinskog učenja dodatno je unapredio funkcionalnost i intuitivnost aplikacija, pružajući korisnicima personalizovano iskustvo koje je prethodnih godina bilo nezamislivo. Ovaj napredak ne samo da je promenio način na koji interagujemo sa našim uređajima, već je i potpuno preoblikovao digitalni pejzaž, otvarajući nove mogućnosti za razvoj i inovacije u budućnosti. Ovaj napredak mobilnih aplikacija pratio je i razvoj tehnologija koje se koriste u izradi istih. Kroz vreme, došli smo do većeg broja tipova mobilnih aplikacija kao i do velikog broja radnih okvira koji nam omogućavaju lakši razvoj i održavanje aplikacija na kojima radimo.

\section{Tipovi mobilnih aplikacija}

Mobilne aplikacije mogu se kategorisati u nekoliko osnovnih tipova, svaki sa svojim specifičnim funkcijama i ciljevima. Delimo ih na nativne aplikacije(eng. native mobile applications), veb aplikacije (eng. web mobile applications), hibridne mobilne aplikacije i progrsivne veb alikacije (eng. progressive web applications - (PWA)). Svaki od ovih tipova ima svoje prednosti i mane, te izbor tipa aplikacije zavisi od specifičnih potreba i ciljeva projekta. U nastavku rada ćemo analizirati svaku od pomenutih kategorija.

\subsection{Nativne mobilne aplikacije}

Nativne mobilne aplikacije (eng. native mobile applications) su programi razvijeni za specifičan operativni sistem, kao što je iOS ili Android, koristeći programerske jezike koji su specifični za svaku od platformi. Tako imamo Swift i Objective C za iOS ili Kotlin i Javu za Android. Ove aplikacije se instaliraju direktno na mobilni uređaj preko prodavnice aplikacija na uređaju i optimizovane su da pružaju maksimalnu efikasnost i iskoristivost hardverskih karakteristika uređaja. Prednsoti ovog tipa mobilnih aplikacija su:

\begin{itemize}
    \item Budući da je ovaj tip aplikacija optimizovan za specifičnu platformu, po pravilu se mogu izvući neuporedivo više performanse u poređenju sa ostalim tipovima.
    \item Fluidne animacije i intuitivan interfejs koji potiče od samog operativnog sistema rezultuju poboljšanim korisničkim iskustvom.
    \item Mogućnost pristupa punom setu hardverskih i softverskih funkcija samog uređaja kao što su kamera, GPS, senzori
    \item Nativne mobilne aplikacije se objavljuju na prodavnicama aplikacija kao sto su Play prodavnica (eng. Play Store) i App prodavnica (eng. App store). Tokom ovog procesa se vrši detaljno ispitivanje aplikacija i na ovaj način se povećava sigurnost.
\end{itemize}
Mane ovog tipa mobilnih aplikacija su:
\begin{itemize}
    \item Razvijanje posebne aplikacija za svaki operativni sistem povećava troškove kao i vreme koje je potrebno da se aplikacija pusti u korišćenje
    \item Različiti operativni sistemi često iziskuju različite programske jezike za razvoj, što može otežati pronalaženje razvojnih timova.
    \item Svaka promena ili ažuriranje aplikacija zahteva odvojeno slanje na odobrenje prodavnicu aplikacija za svaki operativni sistem. Ovaj proces je često vremenski zahtevan.
\end{itemize}

\begin{figure}[h]
    \centering
    \includegraphics[scale=0.5]{docs/images/chapterTwo/nativnaAplikacija.png}
    \caption{Komunikacija nativne aplikacije i operativnog sistema}
    \label{fig:nativnaAplikacija}
\end{figure}

\subsection{Veb mobilne aplikacije}
Veb mobilne aplikacije (eng. web mobile applications) su pristupačne preko internet pretraživača i ne zahtevaju preuzimanje i instalaciju na uređaj kao tradicionalne aplikacije. One su dizajnirane da budu kompatibilne sa različitim platformama i pružaju jedinstveno iskustvo korisnicima na različitim uređajima. One predstavljaju korisnu opciju za aplikacije koje zahtevaju brzu dostupnost i lako održavanje, ali kada je u pitanju duboka integracija sa uređajem i složene interakcije, ovaj tip aplikacije često zaostaje za mogućnostima koje pružaju nativne aplikacije. Prednosti veb mobilnih aplikacija:

\begin{itemize}
    \item Ovaj tip mobilnih aplikacija će raditi na bilo kom uređaju koji ima veb pretraživač. Ova osobina uklanja potrebu za posebnim verzijama za različite operativne sisteme.
    \item Ovaj tip mobilnih aplikacija razvija se samo jednom za sve platforme, čime se troškovi razvoja i ažuriranja značajno smanjuju u odnosu na nativne aplikacije.
    \item Korisnici veb mobilnih aplikacija imaće pristup najnovijim izmenama onog momenta kada te izmene budu puštene na produkciju. Ovo znači da se uklanja potreba za preuzimanjem ažururanja, što znatno olakšava distribuciju i održavanje.
\end{itemize}
Mane ovog tipa mobilnih aplikacija su:
\begin{itemize}
    \item Veb mobilne aplikacije zavise od brzine i kvaliteta internet konekcije, a takođe ne mogu u potpunosti iskoristiti sve hardverske mogućnosti uređaja kao što to mogu nativne mobilne aplikacije.
    \item Korisnički interfejsi ovog tipa mobilnih aplikacija često su manje fluidni i intuitivni u odnosu na nativne mobilne aplikacije, što može uticati na ukupno korisničko iskustvo.
    \item Funkcionalnosti i performanse mogu varirati u zavisnosti od pretraživača koji korisnik upotrebljava, što može rezultovati u nekonzistentnosti u korisničkom iskustvu.
\end{itemize}

\begin{figure}[h]
    \centering
    \includegraphics[scale=0.5]{docs/images/chapterTwo/nativnaAplikacija.png}
    \caption{Komunikacija nativne aplikacije i operativnog sistema}
    \label{fig:nativnaAplikacija}
\end{figure}

% ------------------------------------------------------------------------------
\chapter{Zaključak}
% ------------------------------------------------------------------------------
\pangrami

\pangrami

% ------------------------------------------------------------------------------
% Literatura
% ------------------------------------------------------------------------------
\literatura

% ==============================================================================
% Završni deo teze i prilozi
\backmatter
% ==============================================================================

% ------------------------------------------------------------------------------
% Biografija kandidata
\begin{biografija}
  \textbf{Vuk Stefanović Karadžić} (\emph{Tršić,
    26. oktobar/6. novembar 1787. — Beč, 7. februar 1864.}) bio je
  srpski filolog, reformator srpskog jezika, sakupljač narodnih
  umotvorina i pisac prvog rečnika srpskog jezika.  Vuk je
  najznačajnija ličnost srpske književnosti prve polovine XIX
  veka. Stekao je i nekoliko počasnih mastera.  Učestvovao je u
  Prvom srpskom ustanku kao pisar i činovnik u Negotinskoj krajini, a
  nakon sloma ustanka preselio se u Beč, 1813. godine. Tu je upoznao
  Jerneja Kopitara, cenzora slovenskih knjiga, na čiji je podsticaj
  krenuo u prikupljanje srpskih narodnih pesama, reformu ćirilice i
  borbu za uvođenje narodnog jezika u srpsku književnost. Vukovim
  reformama u srpski jezik je uveden fonetski pravopis, a srpski jezik
  je potisnuo slavenosrpski jezik koji je u to vreme bio jezik
  obrazovanih ljudi. Tako se kao najvažnije godine Vukove reforme
  ističu 1818., 1836., 1839., 1847. i 1852.
\end{biografija}
% ------------------------------------------------------------------------------

\end{document}
